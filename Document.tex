\documentclass[12pt, a4paper]{article}

%------------------------
% IMPORT PACKAGES
%-----------------------
\usepackage[italian,english]{babel}
\usepackage[utf8x]{inputenc}
\usepackage{amsmath,amsthm}
\usepackage{amsfonts}
\usepackage{graphicx}
\usepackage{adjustbox}
\usepackage{float,subfloat}
\usepackage{wrapfig}
\usepackage{lipsum}
\usepackage{bbm}
\usepackage{bm}
\usepackage{empheq}
\usepackage{mathtools}
\usepackage{algorithm,algpseudocode}
\usepackage{booktabs,longtable}
\usepackage{array}
\usepackage{hyperref}
%\usepackage{siunitx}
\usepackage {fancyhdr}
%\sisetup{output-exponent-marker=\ensuremath{\mathrm{e}}}
%\usepackage[acronym,toc,nohypertypes={acronym,notation},nonumberlist,automake]{glossaries}
\usepackage{fancyhdr}
\usepackage[algo2e]{algorithm2e}
\usepackage{cancel}
%\usepackage{appendix}

%\DeclareMathOperator*{\sup}{sup}
\DeclareMathOperator*{\argmax}{arg\,max} % argmax operator

%% remark theoremstyle
%\newtheoremstyle{break}
%{\topsep}{\topsep}%
%{\itshape}{}%
%{\bfseries}{}%
%{\newline}{}%
%\theoremstyle{break}
%\newtheorem{remark}{Remark}[section]

% problem theoremstyle
\newtheoremstyle{problemstyle}  % <name>
{10pt}  % <space above>
{10pt}  % <space below>
{\normalfont} % <body font>
{}  % <indent amount}
{\bfseries\itshape} % <theorem head font>
{\normalfont\bfseries:} % <punctuation after theorem head>
{.5em} % <space after theorem head>
{} % <theorem head spec (can be left empty, meaning `normal')>
\theoremstyle{problemstyle}
\newtheorem{problem}{Problem}[section] % Comment out [section] toremove section number dependence
\newtheorem{definition}{Definition}[section]
\newtheorem{theorem}{Theorem}[section]
\newtheorem{proposition}{Proposition}[section]
\newtheorem{remark}{Remark}[section]
\newtheorem{lemma}{Lemma}[section]
% create command for variance in math mode
\newcommand{\Var}[1]{\operatorname{Var}\left[#1\right]}
\newcommand{\Cov}[1]{\operatorname{Cov}\left[#1\right]}



%opening
\title{Case study answers}
\author{}

\begin{document}

\maketitle

\begin{abstract}
	In this document we provide the answers to the case study problems. Supporting code for numerical results and plots is available at this \href{https://github.com/skiamu/case-study}{github repo}.
\end{abstract}

\section{Question 1}
Let $X \sim \text{logN}\big(\mu, \sigma^2\big)$. Here some standard results about the lognormal distribution which will be used in the following:
\begin{gather}
m =  \mathbb{E}[X]  = \exp\Big(\mu+\sigma^2/2\Big) \\
v=  \Var{X} =  \Big[\exp(\sigma^2-1)\Big]\exp\Big(2\mu+\sigma^2\Big)\\\label{eq:logn_median}
m^{\star} =   \text{Med}[X] = \exp(\mu) 
\end{gather}
\subsection{Question 1.1}
In order compute the median $m^{\star}$ we need to derive the parameters $\mu$ and $\sigma^2$ first. That amounts to solve the following system
\begin{equation}
\begin{cases}
m = \exp\Big(\mu+\sigma^2/2\Big)\\
v = \Big[\exp(\sigma^2-1)\Big]\exp\Big(2\mu+\sigma^2\Big).
\end{cases}
\end{equation}
By noting that the second factor in $v$ is equal to $m^2$ we get that 
\begin{equation}
\begin{cases}
\sigma^2=\log\Big(\frac{v}{m^2}+1\Big)\\
\mu = \log(m) - \sigma^2/2.
\end{cases}
\end{equation}
By plugging the numbers we get $\sigma^2=0.02257$ and $\mu = 0.0469 $. It follows from (\ref{eq:logn_median}) that $m^{\star}=1.0481$.

\subsection{Question 1.2}
\begin{figure}[h]
	\centering
	\includegraphics[scale=0.7]{data/lognorm_hist}
	\caption{Probability density function of $X$ and the histogram of 10000 samples drawn from the distribution of $X$.}
\end{figure}


\section{Question 2}
In this section we answer question 2. Before doing that, we briefly recall the underlying factor model for asset returns in order to set notation 

\begin{definition}
	Let $R$ be an $n$-dimensional random vector describing assets return over a generic period $[t, T]$. We suppose $R$ follows a linear factor model 
	\begin{equation}\label{eq:factor_model}
	R = \beta R_f + \epsilon
	\end{equation}
	where
	\begin{itemize}
		\item $R_f$ is a $f$-dimensional random vector describing factors returns
		\item $\beta \in \mathbb{R}^{n \times f}$ is the factor loadings matrix
		\item $\epsilon$ is the $n$-dimensional random vector of residuals.
	\end{itemize}
The folowing standard hypothesis apply
\begin{itemize}
	\item $\epsilon_i \overset{\mathrm{iid}}{\sim} (0, \Omega_{ii})$, $\forall i = 1, \ldots, n$
	\item $R_f$ and $\epsilon$ are uncorrelated (i.e. $\Cov{R_f, \epsilon}=0_{f\times n}$)
\end{itemize}
\end{definition}

\begin{proposition}
	If $R$ follows the factor model (\ref{eq:factor_model}), then 
	\begin{equation}
	\Cov{R}:=\Sigma_n=\beta\Sigma_f \beta^T + \Omega,
	\end{equation}
	where $\Sigma_f$ is the covariance matrix of $R_f$ and $\Omega$ is the diagonal covariance matrix of $\epsilon$
\end{proposition}
\begin{proof}
	The expected value of $R$ reads $\mathbb{E}[R]=\beta \bar{R}_f$. Therefore
	\begin{align}
	\Cov{R} & = \mathbb{E}\Bigg[\Big(R - \beta \bar{R}_f\Big)\Big(R - \beta \bar{R}_f\Big)^T\Bigg] \\
	& = \mathbb{E}\Bigg[\Big(\beta R_f + \epsilon -  \beta \bar{R}_f\Big)\Big(R_f^T\beta^T + \epsilon^T - \bar{R}_f^T\beta^T\Big)\Bigg]\\ \label{eq:hp1}
	& = \beta\mathbb{E}[R_fR_f^T]\beta^T + \mathbb{E}[\epsilon \epsilon^T] - \beta\bar{R}_f\bar{R}_f^T\beta^T\\ \label{eq:hp2}
	& = \beta\Sigma_f \beta^T + \Omega,
	\end{align}
	where in (\ref{eq:hp1}) we used the fact that $R_f$ and $\epsilon$ are uncorrelated and $\epsilon$ has zero mean.
\end{proof}

\subsection{Question 2.1}
Let $w \in \mathbb{R}^n$ be the vector of portfolio weights. Then
\begin{equation}
\sigma(w)=\sqrt{\Var{w^T R}}=\sqrt{w^T\Sigma_n w} = \sqrt{w^T\beta\Sigma_f\beta^Tw + w^T\Omega w}.
\end{equation}
Plugging the numbers we get $\sigma(w)=40.71\%$

\subsection{Question 2.2}
Let us recall the main results about risk decomposition.
\begin{definition}
	Let $f(x)$ be a continous continuous and differentiable function of $x \in \mathbb{R}^n$. $f$ is said homogenous of degree one if 
	\begin{equation}
	f(cx) = cf(x),
	\end{equation}
	$\forall c \in \mathbf{R}$, $c>0$.
\end{definition}

\begin{theorem}(Euler)
	Let $f(x)$ be a continuous, differentiable and homogenous function of order one. Then 
	\begin{equation}
	f(x) = \sum_{i=1}^{n}x_i \frac{\partial f(x)}{\partial x_i} = x^T \nabla f(x)
	\end{equation}
\end{theorem}


It can be shown that portfolio volatility $\sigma = (w^T\Sigma_n w)^{1/2}$ is an homogenous function of degree one in $w$. Applying Euler's Theorem we have 
\begin{equation}
\sigma(w)=\sum_{i=1}^nw_i \frac{\partial \sigma(w)}{\partial w} = \sum_{i=1}^n w_i \text{MRC}_{i} = \sum_{i=1}^n \text{RC}_{i},
\end{equation}
where MRC stands for \textit{marginal risk contribution} and RC for \textit{risk contribution}. Given that $\nabla\sigma(w) = \frac{\Sigma_n w}{(w^T\Sigma_n w)^{1/2}}$, we have that
\begin{align}
\text{MRC}_i & = \frac{(\Sigma_n w)_i}{(w^T\Sigma_n w)^{1/2}}\\
\text{RC}_i & = w_i \frac{(\Sigma_n w)_i}{(w^T\Sigma_n w)^{1/2}}\\
\text{PRC}_i & = \frac{\text{RC}}{\sigma(w)} = w_i \frac{(\Sigma_n w)_i}{(w^T\Sigma_n w)}
\end{align}

Having this in mind, we can break down portfolio volatility by asset and sectors. Results are shown in Figure (\ref{fig:histo_by_asset}) and (\ref{fig:histo_by_sector}) respectively

\begin{figure}[H]
	\centering
	\includegraphics[scale=0.7]{data/bar_plot_by_asset_Weight_PRC}
	\caption{Weights and percentage risk contributions (PRC) of a portfolio of US stocks}
	\label{fig:histo_by_asset}
\end{figure}
\begin{figure}[H]
 	\centering
 	\includegraphics[scale=0.7]{data/bar_plot_by_sector_Weight_PRC}
 	\caption{Sectors weights and percentage risk contributions (PRC) of a portfolio of US stocks}
 	\label{fig:histo_by_sector}
 \end{figure}
 





\appendix
\section{Appendix}


\end{document}
